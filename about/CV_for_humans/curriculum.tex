% Source code initially created by Philip Empl, sourced from Jan Küster that is open sourced under the MIT License (MIT) 2019.
% Repository: <https://github.com/philipempl/modern-latex-cv>

\documentclass[10pt,A4,english]{article}	


%----------------------------------------------------------------------------------------
%	ENCODING
%----------------------------------------------------------------------------------------

% we use utf8 since we want to build from any machine
\usepackage[utf8]{inputenc}		
\usepackage[USenglish]{isodate}
\usepackage{fancyhdr}
\usepackage[numbers]{natbib}

%----------------------------------------------------------------------------------------
%	LOGIC
%----------------------------------------------------------------------------------------

% provides \isempty test
\usepackage{xstring, xifthen}
\usepackage{enumitem}
\usepackage[english]{babel}
\usepackage{blindtext}
\usepackage{pdfpages}
\usepackage{changepage}

%----------------------------------------------------------------------------------------
%	FONT BASICS
%----------------------------------------------------------------------------------------

% some tex-live fonts - choose your own

%\usepackage[defaultsans]{droidsans}
%\usepackage[default]{comfortaa}
%\usepackage{cmbright}
\usepackage[default]{raleway}
%\usepackage{fetamont}
%\usepackage[default]{gillius}
%\usepackage[light,math]{iwona}
%\usepackage[thin]{roboto} 

% set font default
\renewcommand*\familydefault{\sfdefault} 	
\usepackage[T1]{fontenc}

% more font size definitions
\usepackage{moresize}

%----------------------------------------------------------------------------------------
%	FONT AWESOME ICONS
%---------------------------------------------------------------------------------------- 

% include the fontawesome icon set
\usepackage{fontawesome}

% use to vertically center content
% credits to: http://tex.stackexchange.com/questions/7219/how-to-vertically-center-two-images-next-to-each-other
\newcommand{\vcenteredinclude}[1]{
    \begingroup
        \setbox0=\hbox{\includegraphics{#1}}%
        \parbox{\wd0}{\box0}
    \endgroup
}

\newcommand{\tab}[1]{
    \hspace{.2\textwidth}\rlap{#1}
}

% use to vertically center content
% credits to: http://tex.stackexchange.com/questions/7219/how-to-vertically-center-two-images-next-to-each-other
\newcommand*{\vcenteredhbox}[1]{
    \begingroup
        \setbox0=\hbox{#1}\parbox{\wd0}{\box0}
    \endgroup
}

% icon shortcut
\newcommand{\icon}[3] { 							
	\makebox(#2, #2){\textcolor{maincol}{\csname fa#1\endcsname}}
}	


% icon with text shortcut
\newcommand{\icontext}[4]{ 						
	\vcenteredhbox{\icon{#1}{#2}{#3}}  \hspace{2pt}  \parbox{0.9\mpwidth}{\textcolor{#4}{#3}}
}

% icon with website url
\newcommand{\iconhref}[5]{ 						
    \href{#4}{\vcenteredhbox{\icon{#1}{#2}{#5}}  \hspace{2pt} \textcolor{#5}{#3}}
}

% icon with email link
\newcommand{\iconemail}[5]{ 						
    \href{mailto:#4}{\vcenteredhbox{\icon{#1}{#2}{#5}} \hspace{2pt} {\textcolor{#5}{#3}}}
}

%----------------------------------------------------------------------------------------
%	PAGE LAYOUT  DEFINITIONS
%----------------------------------------------------------------------------------------

% page outer frames (debug-only)
% \usepackage{showframe}		

% we use paracol to display breakable two columns
\usepackage{paracol}
\usepackage{tikzpagenodes}
\usetikzlibrary{calc}
\usepackage{lmodern}
\usepackage{multicol}
\usepackage{lipsum}
\usepackage{atbegshi}
% define page styles using geometry
\usepackage[a4paper]{geometry}

% remove all possible margins
\geometry{top=1cm, bottom=1cm, left=1cm, right=1cm}

\usepackage{fancyhdr}
\pagestyle{empty}

% space between header and content
% \setlength{\headheight}{0pt}

% indentation is zero
\setlength{\parindent}{0mm}

%----------------------------------------------------------------------------------------
%	TABLE /ARRAY DEFINITIONS
%---------------------------------------------------------------------------------------- 

% extended aligning of tabular cells
\usepackage{array}

% custom column right-align with fixed width
% use like p{size} but via x{size}
\newcolumntype{x}[1]{%
>{\raggedleft\hspace{0pt}}p{#1}}%


%----------------------------------------------------------------------------------------
%	GRAPHICS DEFINITIONS
%---------------------------------------------------------------------------------------- 

%for header image
\usepackage{graphicx}

% use this for floating figures
% \usepackage{wrapfig}
% \usepackage{float}
% \floatstyle{boxed} 
% \restylefloat{figure}

%for drawing graphics		
\usepackage{tikz}			
\usepackage{ragged2e}	
\usetikzlibrary{shapes, backgrounds,mindmap, trees}

%----------------------------------------------------------------------------------------
%	Color DEFINITIONS
%---------------------------------------------------------------------------------------- 
\usepackage{transparent}
\usepackage{color}

% primary color
\definecolor{maincol}{RGB}{ 64,64,64}

% accent color, secondary
% \definecolor{accentcol}{RGB}{ 250, 150, 10 }

% dark color
\definecolor{darkcol}{RGB}{ 70, 70, 70 }
% light color
\definecolor{lightcol}{RGB}{245,245,245}
\definecolor{accentcol}{RGB}{59,77,97}
\definecolor{darkblue}{RGB}{111, 143, 175}
\definecolor{lightblue}{RGB}{31, 81, 255}


% Package for links, must be the last package used
%\usepackage{hyperref}

\usepackage[
	citecolor = lightblue,
	anchorcolor = lightblue,
	colorlinks=true,
	linkcolor=lightblue,
	linkbordercolor=lightblue,
	urlcolor=lightblue,
	urlbordercolor = lightblue,
	]{hyperref}

% \makeatletter
% \Hy@AtBeginDocument{
% 	%\def\@pdfborder{0 0 1}
% 	\def\@pdfborderstyle{/S/U/W 1}
% }
% \makeatother

% returns minipage width minus two times \fboxsep
% to keep padding included in width calculations
% can also be used for other boxes / environments
\newcommand{\mpwidth}{\linewidth-\fboxsep-\fboxsep}
	


%============================================================================%
%
%	CV COMMANDS
%
%============================================================================%

%----------------------------------------------------------------------------------------
%	 CV LIST
%----------------------------------------------------------------------------------------

% renders a standard latex list but abstracts away the environment definition (begin/end)
\newcommand{\cvlist}[1] {
	\begin{itemize}{#1}\end{itemize}
}

%----------------------------------------------------------------------------------------
%	 CV TEXT
%----------------------------------------------------------------------------------------

% base class to wrap any text based stuff here. Renders like a paragraph.
% Allows complex commands to be passed, too.
% param 1: *any
\newcommand{\cvtext}[1] {
	\begin{tabular*}{1\mpwidth}{p{0.98\mpwidth}}
		\parbox{1\mpwidth}{#1}
	\end{tabular*}
}
\newcommand{\cvtextsmall}[1] {
	\begin{tabular*}{0.8\mpwidth}{p{0.8\mpwidth}}
		\parbox{0.8\mpwidth}{#1}
	\end{tabular*}
}
%----------------------------------------------------------------------------------------
%	CV SECTION
%----------------------------------------------------------------------------------------

\newlength{\barw}
\newcommand{\cvsection}[1] {
	\vspace{14pt}
	% \settowidth{\barw}{\textbf{\LARGE #1}}
	\cvtext{
		\textbf{\LARGE{\textcolor{darkcol}{#1}}}\\[-4pt]
		\textcolor{accentcol}{ \rule{\barw}{1.5pt} } \\[-5pt]
	}
}

\newcommand{\cvsubsection}[1] {
	\vspace{14pt}
	\settowidth{\barw}{\textbf{\Large #1}}
	\cvtext{
		\textbf{\Large{\textcolor{darkcol}{#1}}}\\[-4pt]
		\textcolor{accentcol}{ \rule{\barw}{1.5pt} } \\
	}
}

\newcommand{\cvheadline}[1] {
	\vspace{16pt}
	\cvtext{
		\textbf{\Huge{\textcolor{accentcol}{#1}}}\\[-4pt]
		 
	}
}

\newcommand{\cvsubheadline}[1] {
	\vspace{16pt}
	\cvtext{
		\textbf{\huge{\textcolor{darkcol}{#1}}}\\[-4pt]
		 
	}
}
%----------------------------------------------------------------------------------------
%	META SKILL
%----------------------------------------------------------------------------------------

\newcommand{\cvskill}[3] {
	\begin{tabular*}{1\mpwidth}{p{0.72\mpwidth}  r}
 		\textcolor{black}{\textbf{#1}} & \textcolor{maincol}{#2}\\
	\end{tabular*}%
	
	\hspace{4pt}
	\begin{tikzpicture}[scale=1,rounded corners=2pt,very thin]
		\fill [lightcol] (0,0) rectangle (1\mpwidth, 0.15);
		\fill [accentcol] (0,0) rectangle (#3\mpwidth, 0.15);
  	\end{tikzpicture}%
}


%----------------------------------------------------------------------------------------
%	 CV EVENT
%----------------------------------------------------------------------------------------

\newcommand{\cvevent}[7] {
	
	% we wrap this part in a parbox, so title and description are not separated on a pagebreak
	% if you need more control on page breaks, remove the parbox
	\parbox{\mpwidth}{
		\begin{tabular*}{1\mpwidth}{p{0.65\mpwidth}  r}
	 		\textcolor{black}{\textbf{#2}} & \colorbox{accentcol}{\makebox[0.35\mpwidth]{\textcolor{white}{\textbf{#1}}}} \\
			\textcolor{maincol}{#3} & \\
		\end{tabular*}\\[8pt]
	
		\ifthenelse{\isempty{#4}}{}{
			\cvtext{#4}\\
		}
	}
	\vspace{14pt}
}


%----------------------------------------------------------------------------------------
%	 CV META EVENT
%----------------------------------------------------------------------------------------

\newcommand{\cvmetaevent}[4] {
	\textcolor{maincol} { \cvtext{\textbf{\begin{flushleft}#1\end{flushleft}}}}

	\ifthenelse{\isempty{#2}}{}{
	\textcolor{black} {\cvtext{\textbf{#2}} }
	}

	\ifthenelse{\isempty{#3}}{}{
		\cvtext{{ \textcolor{maincol} {#3} }}\\
	}

	\cvtext{#4}\\[14pt]
}

%---------------------------------------------------------------------------------------
%	QR CODE
%----------------------------------------------------------------------------------------

% Renders a qrcode image (centered, relative to the parentwidth)
% param 1: percent width, from 0 to 1
\newcommand{\cvqrcode}[1] {
	\begin{center}
		\includegraphics[width={#1}\mpwidth]{qrcode}
	\end{center}
}


% HEADER AND FOOOTER 
%====================================
\newcommand\Header[1]{%
\begin{tikzpicture}[remember picture,overlay]
\fill[accentcol]
  (current page.north west) -- (current page.north east) --
  ([yshift=50pt]current page.north east|-current page text area.north east) --
  ([yshift=50pt,xshift=-3cm]current page.north|-current page text area.north) --
  ([yshift=10pt,xshift=-5cm]current page.north|-current page text area.north) --
  ([yshift=10pt]current page.north west|-current page text area.north west) -- cycle;
\node[font=\sffamily\bfseries\color{white},anchor=west,
  xshift=0.7cm,yshift=-0.32cm] at (current page.north west)
  {\fontsize{12}{12}\selectfont {#1}};
\end{tikzpicture}%
}

\newcommand\Footer[1]{%
\begin{tikzpicture}[remember picture,overlay]
\fill[lightcol]
  (current page.south east) -- (current page.south west) --
  ([yshift=-80pt]current page.south east|-current page text area.south east) --
  ([yshift=-80pt,xshift=-6cm]current page.south|-current page text area.south) --
  ([xshift=-2.5cm,yshift=-10pt]current page.south|-current page text area.south) --	
  ([yshift=-10pt]current page.south east|-current page text area.south east) -- cycle;
\node[yshift=0.32cm,xshift=9cm] at (current page.south) {\fontsize{10}{10}\selectfont \textbf{\thepage}};
\end{tikzpicture}%
}


%=====================================
%============================================================================%
%
%
%
%	DOCUMENT CONTENT
%
%
%
%============================================================================%
\begin{document}

\columnratio{0.31}
\setlength{\columnsep}{2.2em}
\setlength{\columnseprule}{4pt}
\colseprulecolor{white}


% LEBENSLAUF HIERE
\AtBeginShipoutFirst{\Header{CV}\Footer{1}}
\AtBeginShipout{\AtBeginShipoutAddToBox{\Header{CV}\Footer{2}}}

\newpage

\colseprulecolor{lightcol}
\columnratio{0.31}
\setlength{\columnsep}{2.2em}
\setlength{\columnseprule}{4pt}
\begin{paracol}{2}


\begin{leftcolumn}
%---------------------------------------------------------------------------------------
%	META IMAGE
%----------------------------------------------------------------------------------------

% personal picture
\includegraphics[width=\linewidth]{../images/sebastiano.jpg}
% QR code if no picture is allowed
%\includegraphics[width=\linewidth]{../images/qr-code-linkedin.png}

%---------------------------------------------------------------------------------------
%	META SKILLS
%----------------------------------------------------------------------------------------
\fcolorbox{white}{white}{\begin{minipage}[c][1.5cm][c]{1\mpwidth}
	\LARGE{\textbf{\textcolor{maincol}{Sebastiano Ferraris}}} \\[2pt]
	\normalsize{ 
        \textcolor{maincol} {Geospatial Data Scientist, PhD} 
    }
\end{minipage}} \\

% \icontext{CaretRight}{12}{London, since 2014}{black}\\[6pt]
% \icontext{CaretRight}{12}{Italian nationality}{black}\\[6pt]
%\icontext{CaretRight}{12}{unmarried}{black}\\[6pt]

% \cvsection{Contacts}

\icontext{MapMarker}{16}{London}{black}\\[6pt]
% \icontext{MobilePhone}{16}{+44 775689xxxx}{black}\\[6pt]
\iconemail{Envelope}{16}{sebastiano.ferraris@gmail.com}{sebastiano.ferraris@gmail.com}{black}\\[5pt]
\iconhref{Home}{16}{github.io/GeoDsBlog/}{https://sebastianof.github.io/GeoDsBlog/}{black}\\[5pt]
\iconhref{Github}{16}{github.com/SebastianoF}{https://www.github.com/SebastianoF}{black}\\[5pt]
\iconhref{Linkedin}{16}{linkedin.com/in/ibis-redibis/}{https://www.linkedin.com/in/ibis-redibis/}{black}\\[5pt]
\iconhref{Google}{16}{Google Scholar}{https://scholar.google.com/citations?user=1tAeAI0AAAAJ&hl=en}{black}\\[5pt]
\iconhref{GraduationCap}{16}{Research Gate}{https://www.researchgate.net/profile/Sebastiano_Ferraris}{black}\\[5pt]


\cvsection{Skills}

\cvskill{Python} {9+ years} {1} \\[-2pt]

\cvskill{Data science} {9+ years} {1} \\[-2pt]

\cvskill{Algorithms} {9+ years} {1} \\[-2pt]

\cvskill{Artificial intelligence} {4+ years} {0.63} \\[-2pt]

\cvskill{Geospatial data science} {4+ years} {0.5} \\[-2pt]

\cvskill{Medical image analysis} {4 years} {0.45} \\[-2pt]

\cvskill{Discrete events simulation} {1 year} {0.2} \\[-2pt]

\cvskill{Dynamic pricing} {1 year} {0.2} \\[-2pt]



\newpage
%---------------------------------------------------------------------------------------
%	EDUCATION
%----------------------------------------------------------------------------------------
\cvsection{Education}

\cvmetaevent
{2015 - 2018}
{PhD, Centre for Doctoral Training (EPSRC), Medical Imaging}
{University College London}
{\textit{MRI • Pre-clinical studies • Numerical methods for Image registration • 8 Papers published • 12 repositories open sourced}}

\cvmetaevent
{2014 - 2015}
{Master of Research (MRes), Medical Imaging}
{University College London}
{\textit{Numerical methods for image registration • Digital Image Processing • Optics in Medicine}}

\cvmetaevent
{2010 - 2013}
{Master of Science (MSci), Mathematics}
{Università degli studi di Torino}
{\textit{Geometry • Error correcting code theory • Computational modelling}.}

\cvmetaevent
{2006 - 2010}
{Bachelor's of Science (BSc), Mathematics}
{Università degli studi di Torino}
{}

%
%\cvsection{Projekte}

%	\cvlist{
%		\item \hyperlink{https://github.com/philipempl/ether-twin}{\textbf{Ether-Twin.}}\\ Ethereum Applikation für Digital Twins.
%		\item \hyperlink{https://github.com/philipempl/Peter-Pan}{\textbf{Peter Pan.}}\\ Koch-App (t.b.a.).
%		\item \hyperlink{https://github.com/philipempl/Innovation-Tool}{\textbf{Innovation Tool.}}\\ Webcrawler für \hyperlink{https://ibi.de/}{Ibi}.
%		\item \hyperlink{https://github.com/philipempl/cozone}{\textbf{COZONE.}} \\ Soziales Netzwerk (t.b.a.).
%		\item \hyperlink{https://github.com/geritwagner/enlit}{\textbf{ENLIT.}}\\ Exploring new Literature (Bachelorarbeit).
%		\item \textbf{Crowdfunding.} \\Modul mit \hyperlink{https://senacor.com/}{Senacor} für \hyperlink{https://www.paydirekt.de/}{paydirekt}.
%		}

\cvsection{Volunteering}

\icontext{GraduationCap}{12}{Maths Tutor, \href{https://actiontutoring.org.uk/}{Action Tutoring}}{black}\\[6pt]
\icontext{ClockO}{12}{Scanner and Marshall, \href{https://www.parkrun.org.uk/}{Parkrun}}{black}\\[6pt]

\vspace*{0.6cm}
\begin{center}
	\hspace*{-0.8cm}
	\includegraphics[width=3cm]{../images/qr-code-linkedin_70_70_70.png}	
\end{center}


	
%\cvqrcode{0.3}

\end{leftcolumn}
\begin{rightcolumn}
%---------------------------------------------------------------------------------------
%	TITLE  HEADER
%----------------------------------------------------------------------------------------


%---------------------------------------------------------------------------------------
%	PROFILE
%----------------------------------------------------------------------------------------
\cvsection{Summary}
~
% \vspace{4pt}

5+ years of experience in developing prototypes and algorithms, from proof of concept to production. Proven track records of implementing, validating, and scaling algorithms to solve a range of research and industrial problems. Keen on addressing the challenges around productionisation and algorithms continuous performance validation.  Scientific author published in international journals.

%---------------------------------------------------------------------------------------
%	WORK EXPERIENCE
%----------------------------------------------------------------------------------------

% \vspace{5pt}
\cvsection{Experience}
% \vspace{4pt}



\cvevent
{June 2020 - June 2024}
	{Data scientist | \href{https://www.generalsystem.com/}{General System}}
	{
		Startup in stealth model until 2022. High performant real-time analytic platform for high volume (100+Bn) spatiotemporal data \emph{Startup declared insolvency in June 2024}.
	}
	{
		• Wrote and tested production code for a novel, robust and linear-time {\bf clustering algorithm} to detect dwells in mobility data in Python, scikit-learn, pandas, numpy, streamlit, DeckGl, \href{https://kepler.gl/}{KeplerGl}.\\
		• Developed a {\bf hierarchical density based algorithm} prototype for spatiotemporal data. \\
		• Created on-line and batch outlier detections and corrections algorithms for spatiotemporal data. \\
		• Researched and prototyped two linear-time {\bf data fusion algorithms} for detecting co-locations across multiple layers, such as AdTech, AIS and ADS-B datasets.\\
		• Leveraged these algorithms to detect: dark vessels, crowds gathering, consumers' patterns, and cross visitations, setting up data processing pipelines with {\bf Databricks} and the internally developed {\bf Data Flow Index}. \\
		• Worked closely with the Front and Back End production teams to turn prototypes into scaled up products, with {\bf AGILE} and twelve factor app methodology, with {\bf CI/CD}, {\bf Open API}, {\bf unittesting}, {\bf integration testing}, {\bf contract testing}. \\
		• Developed and \href{https://github.com/thegeneralsystem}{open sourced} a python library to provide users tooling and examples for the \href{https://www.generalsystem.com/product}{Data Flow Index}.\\
		• Contributed to the \href{https://www.generalsystem.com/blog}{company blog} aimed at building a community around the hot topics of spatiotemporal data science. \\
		• Supported customer success and marketing supporting the creation of visualizations and presentation materials. \\
	}
\vfill\null


\cvevent
{Sept 2019 - June 2020}
	{Algorithm engineer | \href{https://www.paceup.com/}{Pace}}
	{
		Predictive analytics and dynamic pricing for the hospitality industry integrated in the PMS.
		\emph{Left due to COVID-19 disruption in the hospitality industry}
	}
	{
		•  Participated in developing an {\bf agent based simulation} aimed at validating the core prediction algorithm. \\
		• Maintained Python and SQLAlchemy production code with the Back End team.\\
		• Migrated production codebase to {\bf Dask} to improve scalability.
	}
\vfill\null

	
\cvevent
{Oct 2018 - June 2019}
	{Back end developer | \href{https://www.thoughtmachine.net/}{Thought Machine}}
	{
		Cloud native core banking.
		\emph{Left to pursue a career in algorithms development and data science more aligned with my studies.}
	}
	{
		• Member of the corporate infrastructure team aimed at developing the tools to enable deployment, testing and integration to increase developers speed.\\
		• Contributed writing and improving the internal Python CLI to release and cloud deployment. \\
		• Wrote and managed jenkins deployment cron jobs.\\
		• Wrote a Python service to scrape Phabricator and sync its tickets into JIRA.\\
	}
\vfill\null

\cvevent
{Sept 2014 - Sept 2018}
	{MRes + PhD in medical image analysis | \href{https://www.ucl.ac.uk/medical-physics-biomedical-engineering/study/postgraduate-research/medical-imaging-mres-mphilphd}{UCL} }
	{Research Student}
	{
		• Implemented ML models and automated statistical analysis pipelines to quantify the negative effects of steroids administration in preterm birth, as part of a multi-disciplinary international research team. \\
		• Developed a novel numerical analysis method to integrate ODE in {\bf diffeomorphic image registration}. \\
		• Published \href{https://scholar.google.com/citations?user=1tAeAI0AAAAJ&hl=en}{7 peer reviewed papers} also on \href{https://www.sciencedirect.com/science/article/pii/S1053811918305366?via\%3Dihub}{Neuroimage} 
		and \href{https://www.nature.com/articles/s41598-019-39922-8}{Nature Scientific Report} about \href{https://www.cv-foundation.org/openaccess/content_cvpr_2016_workshops/w15/papers/Ferraris_Accurate_Small_Deformation_CVPR_2016_paper.pdf}{diffeomorphic image registration} and \href{https://www.sciencedirect.com/science/article/pii/S1053811918305366?via\%3Dihub}{Machine Learning for automated MRI segmentation}.\\
		• Reproducible research advocate: open sourced 12 Python libraries (\href{https://discovery.ucl.ac.uk/id/eprint/10072833/}{Sec 7.2.2 of my PhD Thesis}), and one \href{https://zenodo.org/record/1289776}{micro MRI dataset}.\\
	}
\vfill\null


\cvevent
{March 2013 - June 2014}
	{Industrial simulation modeller | \href{https://www.simtec-group.eu/it/}{SimTec} | Turin, IT }
	{Automotive industry, discrete events simulation}
	{
		• Developed material flow simulation models with PlantSimulation (SimTalk) to estimate efficiency, remove bottlenecks and dimension buffers. \\
		• Supported industrial plant layout design for a range of clients in Italy and Germany. \\
		•  Developed in-house shortest path algorithms for the internal and external logistics of assembly parts, to reduce lags  in JIT manufacturing. \\
		• Presented my results at the first annual Tecnomatix Plant Simulation User Conference in Stuttgart.\\
	}
\vfill\null


\cvevent
{June 2011 - Oct 2011}
	{Developer | \href{https://www.tc-web.it}{TcWeb} | Turin, IT }
	{Web development and technology consulting}
	{
		• Junior developer, Java J2EE and Struts 2 for developing the website of Regione Piemonte.\\
		• Document existing codebase with UML diagrams. \\
		• Prototyped and implemented a generalized version of the Hungarian Algorithm to digitalise newspaper pages, reducing 2 months of manual work in less than 1 minutes of computations.\\
	}
\vfill\null


\cvsection{Selected publications}

\begin{itemize}[leftmargin=*]
\item Ferraris S, et al. 
\href{https://www.ncbi.nlm.nih.gov/pmc/articles/PMC6203700/}{"A magnetic resonance multi-atlas for the neonatal rabbit brain"}.
Neuroimage. \textit{Neuroimage} 2018 Oct doi: 10.1016/j.neuroimage.2018.06.029.
\item Ferraris S, et al. 
\href{https://ieeexplore.ieee.org/document/7789554}{"Accurate small deformation exponential approximant to integrate large velocity fields: Application to image registration"}. 
In: \textit{Proceedings of the IEEE Conference on Computer Vision and Pattern Recognition}, Lipsum, June 12-17, 2020.
\item Ferraris S, et al.
\href{https://joss.theoj.org/papers/10.21105/joss.00354}{"Bruker2nifti: Magnetic resonance images converter from bruker ParaVision to NIfTI format"}. 
In: \textit{Journal of Open Source Software}, 2017.
\item Ferraris S 
\href{https://discovery.ucl.ac.uk/id/eprint/10072833/}{"Image computing tools for the investigation of the neurological effects of preterm birth and corticosteroid administration"} 
\textit{PhD thesis, University College London}, 2019.


\end{itemize}
\mbox{}
\vfill
% Please see my \href{https://scholar.google.com/citations?user=1tAeAI0AAAAJ&hl=en}{Google Scholar Profile} for the complete list of publications.
% \mbox{}
% \vfill
% \mbox{}
% \vfill
% \mbox{}
% \vfill
% \mbox{}
% \vfill
% \mbox{}
% \vfill
% \mbox{}
% \vfill
% \mbox{}
% \vfill
% \mbox{}
% \vfill
% \mbox{}
% \vfill
% \mbox{}
% \vfill
% \mbox{}
% \vfill
% \mbox{}
% \vfill
% \mbox{}
% \vfill
% \mbox{}
% \vfill
% \mbox{}


\today, \href{https://sebastianof.github.io/GeoDsBlog/about/CV_modern/curriculum.pdf}{CV for humans}. \hspace{1cm}   \hrulefill

% \hspace*{30mm}\phantom{Lorem, \today }~~~~~~~~~~~~~~~~~~~~~~~~~~~Sebastiano Ferraris

\end{rightcolumn}
\end{paracol}


\end{document}
